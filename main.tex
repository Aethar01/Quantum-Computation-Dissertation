\documentclass[12pt]{amsart}
\usepackage{amsmath,amssymb,amsthm,amsfonts}
\usepackage{graphicx}
\usepackage{hyperref}
\usepackage{babel}
\usepackage{qcircuit}
\usepackage{datetime}
\let\oldtocsection=\tocsection

\let\oldtocsubsection=\tocsubsection

\let\oldtocsubsubsection=\tocsubsubsection

\renewcommand{\tocsection}[2]{\hspace{0em}\oldtocsection{#1}{#2}}
\renewcommand{\tocsubsection}[2]{\hspace{1em}\oldtocsubsection{#1}{#2}}
\renewcommand{\tocsubsubsection}[2]{\hspace{2em}\oldtocsubsubsection{#1}{#2}}
\makeatletter
\renewcommand\subsection{\@startsection{subsection}{2}%
  \z@{.5\linespacing\@plus.7\linespacing}{-.5em}%
  {\normalfont\scshape}}
\makeatother

\newdateformat{monthyeardate}{%
  \monthname[\THEMONTH] \THEYEAR}

\title{Quantum Computing}
\author{Elliott Ashby \\ Physics and Astronomy \\ University of Southampton}
\date{\monthyeardate\today}
% ----------------------------------------------------------------

\begin{document}
\begin{abstract}
    Placeholder for abstract.
\end{abstract}
\maketitle
\tableofcontents
\section{Introduction}
Animals gain advantages in many ways, one of which is the exploitation of properties of the physical world. This has come to culmination in humans; In 1941 we saw the creation of the first programmable computer, the Z3, by Konrad Zuse, and in the following decades we have continually perfected this technology. The modern computer that we use today is, at its fundamental principals, identical to the Z3, performing binary operations on "bits" - a 1 or a 0 - of data in order to encode useful computational results. \\
\\
The Z3 used 

\section{An Overview of Key Concepts}
\subsection{History of Quantum Computing}
\subsection{Limitations of Classical Computers and the Need for Quantum Computing}
\subsection{Quantum Bits and Parrallelism}
\subsection{Quantum Superposition and Entanglement}
\subsection{The Thermodynamics of Quantum Computing}
\subsection{Quantum Algorithms}
\subsection{Quantum Error Correction}
\subsection{Experimental Quantum Computing}




\end{document}
